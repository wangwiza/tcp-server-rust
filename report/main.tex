\documentclass[11pt]{article}
\usepackage[a4paper, total={18cm, 26cm}]{geometry}
\linespread{1.15}

\title{CS3211 Assignment 3}
\author{David Zhu (E0958755), William Zi Ang Wang (E1496974)}
\date{AY24/25 Semester 2}

% set font
\usepackage[T1]{fontenc}
\usepackage[utf8]{inputenc}
% \usepackage{mathpazo}
\usepackage{libertine}
% \usepackage{helvet}
% \renewcommand{\familydefault}{\sfdefault}

\usepackage{xcolor}
\usepackage{hyperref,theoremref}
\usepackage[parfill]{parskip}
\hypersetup{
    colorlinks=true, %set true if you want colored links
    linktoc=all,     %set to all if you want both sections and subsections linked
    linkcolor=blue,  %choose some color if you want links to stand out
    citecolor=black,
    filecolor=black,
    linkcolor=black,
    urlcolor=black,
    bookmarksnumbered=true,
    bookmarksopen=true
}

\usepackage{graphicx}
\graphicspath{ {images/} }

\usepackage{minted}
\usepackage{mdframed}
\definecolor{bg}{rgb}{0.95,0.95,0.95}
\usepackage{enumitem}

\usepackage{amssymb,amsmath,amsthm,amsfonts}
\usepackage{bm}
\usepackage{mathtools}
\usepackage{multicol,multirow}

\usepackage{varwidth}

% for augmented matrix
\makeatletter
\renewcommand*\env@matrix[1][*\c@MaxMatrixCols c]{%
  \hskip -\arraycolsep
  \let\@ifnextchar\new@ifnextchar
  \array{#1}}
\makeatother

% make all sub items use bullets as well
\setlist[itemize]{label=$\bullet$}

\usepackage{listings}

% Setup the listings package
\lstset{
  basicstyle=\small\ttfamily, % The style of the code
  numbers=left,         % Where to put the line numbers
  numbersep=5pt,        % How far the line numbers are from the code
  tabsize=2,            % Sets default tab size to 4 spaces
  extendedchars=true,
  breaklines=true,      % Sets automatic line breaking
  frame=single,         % Adds a frame around the code
  breakatwhitespace=true
}
\usepackage{tikz}
\usetikzlibrary{tikzmark, arrows.meta, positioning}

% numbered list with only numbers for sub items
\newlist{numlist}{enumerate}{10}
\setlist[numlist]{label*=\arabic*.}



\begin{document}

\maketitle

% Page Limit is 2

\section{TCP Server}

% - An outline and brief explanation of your TCP server - include all assumptions, as well as any non-trivial implementation details related to processing requests from multiple clients.
Our TCP server utilizes the \texttt{tokio} library for non-blocking I/O operations.

\section{Concurrency Paradigm}

% - Explain the concurrency paradigm used in your concurrent implementation and how the clients' requests are being handled concurrently.
We used a couple different concurrency paradigm/techniques:
\begin{itemize}
    \item We worked with asynchronous programming by turning the connection handling and value parsing into async functions. A tokio thread is spawned for every connection.
    \item We also used thread pools using both tokio and rayon to allow I/O and CPU task parallelism.
    \item We use an ARC pointer to pass the rayon pool down the async functions `handle_connection` and `get_task_value` so that rayon threads may be spawned for CPU tasks.
    \item We use message passing using a one-shot channel to bridge rayon to tokio.
\end{itemize}


\section{Concurrency Level}

% - What level of concurrency does your server achieve and why? Explain your design. Briefly explain ALL cases when the concurrency level decreases from your claimed level of concurrency.
% 1: No Concurrency 	The server handles one client at a time and one request from the clients at a time
% 2: Client-Level Concurrency 	Multiple clients, each with its own TCP stream, can run concurrently, but I/O and CPU tasks in the same CPU are not executed concurrently
% 3: Task-Level Concurrency 	Multiple clients can run concurrently + I/O and CPU tasks in the same CPU are executed concurrently.
Our server achieves task-level concurrency.
\begin{enumerate}
    \item Multiple clients can run concurrently. Each connection is wrapped in tokio::spawn, and tokio schedules these tasks across its 6 worker threads. 
    \item I/O and CPU tasks can execute concurrently. I/O tasks run on tokio's async threads, while CPU tasks are offloaded to rayon's 10-thread pool
\end{enumerate}

\section{Server Parallel Tasks}

% - Will your server run tasks in parallel? Briefly explain.
Our server will run tasks in parallel.
\begin{itemize}
    \item CPU-intensive tasks are distributed across rayon threads, so two CPU tasks from different clients execute in parallel.
    \item I/O tasks are async and concurrent on tokio threads. Multiple tokio threads can handle separate I/O tasks in parallel.
\end{itemize}
However, tasks within a single client are processed sequentially as the protocol enforces line-by-line processing.

% - If you have tried multiple implementations, explain the differences and your evolution to the current submission. There is no need to submit your alternative implementations.
\begin{itemize}
    \item Initial solution: uses tokio, runs async tasks asynchronously, and cpu tasks on its own thread using tokio's spawn_blocking. However, tokio docs suggest using rayon for CPU tasks.
    \item Final solution: as described in this report, uses rayon thread pool. This also allows to avoids the runtime error of spawning too many tokio threads too by the CPU tasks.
\end{itemize}

\end{document}

